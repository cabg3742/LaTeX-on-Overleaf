\newglossaryentry{argument}{
    name        = {argument},
    description = {(\textit{Programming}) The actual information stored in a \gls{variable}.}
}

\newglossaryentry{bookmark}{
    name        = {bookmark},
    description = {(\textit{\LaTeX}) A marker used to identify a section, a figure, a table, and more inside a document. It comes before the description of said object or place, such as the numbered marker before a section title. It can be created with numbers, letters, or a combination of both.}
}

\newglossaryentry{citation}{
    name        = {citation},
    description = {Similar to a \gls{reference}, but specific to bibliographical \glspl{source}.}
}

\newglossaryentry{class}{
    name        = {class},
    plural      = {classes},
    description = {(\textit{\LaTeX}) A kind of \gls{macro} that defines a document layout standard.}
}

\newglossaryentry{command}{
    name        = {command},
    description = {(\textit{\LaTeX}) A kind of \gls{macro} that reserves special words to define and alter a document's behavior.}
}

\newglossaryentry{command-prog}{
    name        = {command},
    description = {(\textit{Programming}) A code directive to perform a certain task.},
    parent      = {command}
}

\newglossaryentry{conditional}{
    name        = {conditional},
    description = {(\textit{\LaTeX}) A kind of \gls{macro} that handles decisions.}
}

\newglossaryentry{definition}{
    name        = {definition},
    description = {(\textit{\LaTeX}) A kind of \gls{command} that uses \TeX{} \glspl{primitive}.}
}

\newglossaryentry{environment}{
    name        = {environment},
    description = {(\textit{\LaTeX}) A kind of \gls{macro} that defines a restricted scope where only specific behaviors are allowed.}
}

\newglossaryentry{group}{
    name        = {group},
    description = {(\textit{\LaTeX}) The restricted scope of an \gls{environment}.}
}

\newglossaryentry{hook}{
    name        = {hook},
    description = {(\textit{\LaTeX}) A kind of \gls{macro} that defines special behaviors when intercepting code calls, events or messages.}
}

\newglossaryentry{hyperlink}{
    name        = {hyperlink},
    description = {A link from a hypertext file or document to another location or file, typically activated by clicking on a highlighted word or image on the screen. Hyperlinks can be found in \glspl{citation}, \glspl{URL}, general \glspl{reference}, acronyms, etc.}
}

\newglossaryentry{key}{
    name        = {key},
    description = {See \glsseelist{parameter}.}
}

\newglossaryentry{label}{
    name        = {label},
    description = {(\textit{\LaTeX}) The \gls{argument} used by a \glsdisp{reference}{referencing} \gls{command} to establish the link between an object and the place it's being called from.}
}

\newglossaryentry{loop}{
    name        = {loop},
    description = {(\textit{\LaTeX}) A kind of \gls{macro} that handles repeated sequences of code until met a certain \gls{conditional} statement is met.}
}

\newglossaryentry{macro}{
    name        = {macro},
    description = {(\textit{\LaTeX}) Any set of code that can be declared and called inside a document. It encapsulates \glspl{class}, \glspl{package}, \glspl{environment}, \glspl{command}, \glspl{conditional}, \glspl{loop}, \glspl{hook}, \glspl{variable}, and more. It can be thought of as shorthand code that abbreviates a more complicated sequence of code.}
}

\newglossaryentry{package}{
    name        = {package},
    description = {(\textit{\LaTeX}) A kind of \gls{macro} that defines stylistic attributes and enhancements to the capabilities of a document.}
}

\newglossaryentry{parameter}{
    name        = {parameter},
    description = {(\textit{Programmming}) A kind of \gls{variable} used as input to pass information to other parts of the code, across scopes.}
}

\newglossaryentry{primitive}{
    name        = {primitive},
    description = {(\textit{\LaTeX}) A built-in, building-block \gls{macro} of the \TeX{} engine.}
}

\newglossaryentry{reference}{
    name        = {reference},
    description = {A specific mention that brings attention to a specific object in a document. Referencing in \LaTeX{} is provided with \glspl{hyperlink}.}
}

\newglossaryentry{source}{
    name        = {source},
    description = {A document or work that serves as the basis, proves, supports or supplements a piece of information.}
}

\newglossaryentry{token}{
    name        = {token},
    description = {(\textit{\LaTeX}) A special calculated integer used by the \TeX{} engine to identify control sequences or string characters.}
}

\newglossaryentry{value}{
    name        = {value},
    description = {See \glsseelist{argument}.}
}

\newglossaryentry{variable}{
    name        = {variable},
    description = {(\textit{Programming}) Stored information with symbolic representation.}
}