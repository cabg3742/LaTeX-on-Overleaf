The \texttt{lstlisting} \gls{environment} is a kind of extensible / configurable typesetting \gls{environment}, much like a supercharged version of \texttt{verbatim}:
\begin{itemize}
    \item it displays code as-is;
    \item it provides \glsdisp{label}{labelling} of code chunks with the \texttt{\textbackslash{}caption\{[]\}} / \texttt{\textbackslash{}label\{\}} (see \S~\ref{sec:tutorial/latex/crossref/label}). Notice the difference between the caption \gls{command} in \S~\ref{sec:tutorial/latex/fig} or \S~\ref{sec:tutorial/latex/table/cmds} and this one: the name for the list of listings must be declared inside the braces, with the name under the listing;
    \item it provides syntax highlighting for a variety of programming languages. One can even set their own language.
\end{itemize}

The \gls{environment} is set with \verb"\begin{lstlisting}[]". Options such as language, caption, and \gls{label} are set between the brackets.
\bigskip

\begin{lstlisting}[
    language   = LaTeX,
    caption    = {[\LaTeX{} listing \glsentrytext{environment}]\LaTeX{} listing \gls{environment}.},
    label      = {lst:tutorial/latex/listing},
    mathescape = true
]
\begin{lstlisting}[
    language = LaTeX,
    caption  = {[\LaTeX{} listing \glsentrytext{environment}]\LaTeX{} listing \gls{environment}.},
    label    = {lst:tutorial/latex/listing}
]
...
\$$end{lstlisting}
\end{lstlisting}