Standard column types have a pretty limited usage: they either don't allow text wrapping and/or don't provide vertical alignment and/or don't provide horizontal alignment. That's why I created nine new column types, that provide every type of vertical or horizontal alignment while all supporting text wrapping. They all have width as an \gls{argument}.

\begingroup
    \setlength{\columnA}{\dimexpr .25\linewidth}
    \setlength{\columnB}{\dimexpr \linewidth-\columnA}
    
    \setlength{\columnA}{\columnA-2\tabcolsep-3\vbar/2}
    \setlength{\columnB}{\columnB-2\tabcolsep-3\vbar/2}
    
    \begin{longtable}%
        {|\CC{\columnA}|%
          \LC{\columnB}|%
        }
        \caption[Custom table columns]{Custom table columns.}%
        \label{tab:tutorial/latex/custom/table}\\
        
        \hline
        \textbf{\Gls{command}}
            &\multicolumn{1}{c|}{\textbf{Description}}
        \\\hline
        \endfirsthead
        
        \hline
        \textbf{\Gls{command}}
            &\multicolumn{1}{c|}{\textbf{Description}}
        \\\hline
        \endhead
        
        \texttt{\textbackslash{}LT\{\}}
            &Left-aligned horizontally, top-aligned vertically.
        \\\hline
        
        \texttt{\textbackslash{}CT\{\}}
            &Centered horizontally, top-aligned vertically.
        \\\hline
        
        \texttt{\textbackslash{}RT\{\}}
            &Right-aligned horizontally, top-aligned vertically.
        \\\hline
        
        \texttt{\textbackslash{}LC\{\}}
            &Left-aligned horizontally, centered vertically.
        \\\hline
        
        \texttt{\textbackslash{}CC\{\}}
            &Centered horizontally, centered vertically.
        \\\hline
        
        \texttt{\textbackslash{}RC\{\}}
            &Right-aligned horizontally, centered vertically.
        \\\hline
        
        \texttt{\textbackslash{}LB\{\}}
            &Left-aligned horizontally, bottom-aligned vertically.
        \\\hline
        
        \texttt{\textbackslash{}CB\{\}}
            &Centered horizontally, bottom-aligned vertically.
        \\\hline
        
        \texttt{\textbackslash{}RB\{\}}
            &Right-aligned horizontally, bottom-aligned vertically.
        \\\hline
    \end{longtable}
\endgroup

Column width can also be finicky. For instance, if we want the table to be \textit{exactly} the width of the text area, we need to take the table column separation and the vertical rule thickness into account. These \glspl{value} can all be expressed as a ratio of \texttt{\textbackslash{}linewidth}, $k\in\{0,1\}$:

\begin{equation}
    1 
        = \sum\limits_{n=1}^{N} k_n + 2Nk_{\,\text{tabcolsep}} + M_{\,|}k_{\,|} + M_{\,||}k_{\,||}
    \label{eq:tutorial/latex/custom/table/width}
\end{equation}

\begin{itemize}
    \item $k_n$ is the thickness of each column;
    \item $k_{\,\text{tabcolsep}}$ is the thickness of the tabular separation between columns. It appears twice between columns and once on each side of the table;
    \item $k_{\,|}$ is the thickness of a singular vertical bar, defined as \texttt{\textbackslash{}vbar};
    \item $k_{\,||}$ is the thickness of a double vertical bar, defined as \texttt{\textbackslash{}doublevbar};
    \item $M_{\,|}$ is the number of single vertical bars;
    \item $M_{\,||}$ is the number of double vertical bars;
    \item $N$ is the number of columns.
\end{itemize}

For our next table, we must redefine columns so the sum of all elements in equation \ref{eq:tutorial/latex/custom/table/width} is always equal to \texttt{\textbackslash{}linewidth}. To do this, we can first start splitting our columns as if table column separation and vertical bars didn't exist, like so:
\bigskip

\begin{lstlisting}[%
    language = {LaTeX},
    caption  = {[Column length assignment (I)]Column length assignment (I).},
    label    = {lst:tutorial/latex/custom/table/column1}
]
\setlength{\columnA}{16\linewidth\31}
\setlength{\columnB}{8\linewidth\31}
\setlength{\columnC}{4\linewidth\31}
\setlength{\columnD}{2\linewidth\31}
\setlength{\columnE}{\linewidth-\columnA-\columnB-\columnC-\columnD}
\end{lstlisting}

Then, we need to account for the dimensions we just ignored. Every column as padding on both sides, so this one is easy. Vertical bars are trickier: assuming we separate every column with vertical bars, and our table has a frame, we have one more bar than there are columns. We can then split our total bar thickness by the number of columns for each column. If there were a mix of single and double vertical bars, same principle applies: divide the number of single bars by the number of columns for each, and the number of double bars by the numbers of columns for each:
\bigskip

\begin{lstlisting}[%
    language = LaTeX,
    caption  = {[Column length assignment (II)]Column length assignment (II).},
    label    = {lst:tutorial/latex/custom/table/column2}
]
\setlength{\columnA}{\columnA-2\tabcolsep-6\vbar/5-0\doublevbar/5}
\setlength{\columnB}{\columnB-2\tabcolsep-6\vbar/5-0\doublevbar/5}
\setlength{\columnC}{\columnC-2\tabcolsep-6\vbar/5-0\doublevbar/5}
\setlength{\columnD}{\columnD-2\tabcolsep-6\vbar/5-0\doublevbar/5}
\setlength{\columnE}{\columnE-2\tabcolsep-6\vbar/5-0\doublevbar/5}
\end{lstlisting}

If we use \texttt{\textbackslash{}multicolumn\{\}\{\}\{\}} later inside the table, we may not need to specify the width: if we want it to use all the available space, we can specify the default column types \texttt{l}, \texttt{c}, and \texttt{r} (see table \ref{tab:tutorial/latex/table/column}). However, if we need to specify the vertical alignment, we will need to use custom column types (table \ref{tab:tutorial/latex/custom/table}) that take width as an argument. To do this, you need to assign it the sum of all column widths + whatever there is in between. Continuing further with our example:
\bigskip

\begin{lstlisting}[%
    language = LaTeX,
    caption  = {[Multi-column with custom column widths]Multi-column with custom column widths.},
    label    = {lst:tutorial/latex/custom/table/multicolumn}
]
\begin{longtable}%
    {|\RC{\columnA}|% Right  + center alignment
      \CC{\columnB}|% Center + center alignment
      \CC{\columnC}|% Center + center alignment
      \CC{\columnD}|% Center + center alignment
      \LC{\columnE}|% Left   + center alignment
    }
    ...
    
    % First header
    \hline
    \multicolumn{1}{|\CT{\columnA}|}{Column title \#1}
        & \multicolumn{3}{\CT{\columnB+\columnC+\columnD+4\tabcolsep+2\vbar}}{Column title \#2}
        & \multicolumn{1}{\CT{\columnE}|}{Column title \#3}
    \\\hline
    \endfirsthead
    ...
\end{lstlisting}