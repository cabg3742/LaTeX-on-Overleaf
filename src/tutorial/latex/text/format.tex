\noindent Text formatting can be split into three categories:
\begin{itemize}
    \item family:~defines the font type;
    \item series:~defines the line thickness;
    \item shape:~defines the form of the letters.
\end{itemize}

Note that \texttt{\underline{\textit{\textsc{\textbf{you can combine them all as you wish}}}}}, though the result might not be aesthetically pleasing. Also note that not all fonts support the various options, and some of them are not orthogonal within a font: some options will conflict and the resulting text won't look like you expected. Tables \ref{tab:tutorial/latex/text/format/family} to \ref{tab:tutorial/latex/text/format/other} document the most common text \glspl{command}. 

\begingroup
    \setlength{\columnA}{\dimexpr .30\linewidth}
    \setlength{\columnB}{\dimexpr .13\linewidth}
    \setlength{\columnC}{\dimexpr \linewidth-\columnA-\columnB}
    
    \setlength{\columnA}{\columnA-2\tabcolsep-4\vbar/3}
    \setlength{\columnB}{\columnB-2\tabcolsep-4\vbar/3}
    \setlength{\columnC}{\columnC-2\tabcolsep-4\vbar/3}
    
    \begin{longtable}%
        {|\CC{\columnA}|%
          \CC{\columnB}|%
          \LC{\columnC}|%
        }
        \caption[\LaTeX{} text formatting (family)]{\LaTeX{} text formatting (family).}%
        \label{tab:tutorial/latex/text/format/family}\\
        
        \hline
        \multicolumn{1}{|\CC{\columnA}|}{\textbf{\Gls{command}}}
            &\multicolumn{1}{\CC{\columnB}|}{\textbf{Example}}
            &\multicolumn{1}{\CC{\columnC}|}{\textbf{Description}}
        \\\hline
        \endfirsthead
        
        \hline
        \multicolumn{1}{|\CC{\columnA}|}{\textbf{\Gls{command}}}
            &\multicolumn{1}{\CC{\columnB}|}{\textbf{Example}}
            &\multicolumn{1}{\CC{\columnC}|}{\textbf{Description}}
        \\\hline
        \endhead
        
        \texttt{\textbackslash{}textrm\{\}}
            &\textrm{Text}
            &Roman text.
        \\\hline
        
        \texttt{\textbackslash{}textsf\{\}}
            &\textsf{Text}
            &Sans serif text.
        \\\hline
        
        \texttt{\textbackslash{}texttt\{\}}
            &\texttt{Text}
            &Typewriter / code text.
        \\\hline
    \end{longtable}
\endgroup

\begingroup
    \setlength{\columnA}{\dimexpr .30\linewidth}
    \setlength{\columnB}{\dimexpr .13\linewidth}
    \setlength{\columnC}{\dimexpr \linewidth-\columnA-\columnB}
    
    \setlength{\columnA}{\columnA-2\tabcolsep-4\vbar/3}
    \setlength{\columnB}{\columnB-2\tabcolsep-4\vbar/3}
    \setlength{\columnC}{\columnC-2\tabcolsep-4\vbar/3}
    
    \begin{longtable}%
        {|\CC{\columnA}|%
          \CC{\columnB}|%
          \LC{\columnC}|%
        }
        \caption[\LaTeX{} text formatting (series)]{\LaTeX{} text formatting (series).}%
        \label{tab:tutorial/latex/text/format/series}\\
        
        \hline
        \multicolumn{1}{|\CC{\columnA}|}{\textbf{\Gls{command}}}
            &\multicolumn{1}{\CC{\columnB}|}{\textbf{Example}}
            &\multicolumn{1}{\CC{\columnC}|}{\textbf{Description}}
        \\\hline
        \endfirsthead
        
        \hline
        \multicolumn{1}{|\CC{\columnA}|}{\textbf{\Gls{command}}}
            &\multicolumn{1}{\CC{\columnB}|}{\textbf{Example}}
            &\multicolumn{1}{\CC{\columnC}|}{\textbf{Description}}
        \\\hline
        \endhead
        
        \texttt{\textbackslash{}textbf\{\}}
            &\textbf{Text}
            &Bold text.
             \newline -- Shortcut: {\small \tcbox{\texttt{Ctrl}} + \tcbox{\texttt{B}}}
        \\\hline
        
        \texttt{\textbackslash{}textmd\{\}}
            &\textmd{Text}
            &Medium text.
        \\\hline
    \end{longtable}
\endgroup

\begingroup
    \setlength{\columnA}{\dimexpr .30\linewidth}
    \setlength{\columnB}{\dimexpr .13\linewidth}
    \setlength{\columnC}{\dimexpr \linewidth-\columnA-\columnB}
    
    \setlength{\columnA}{\columnA-2\tabcolsep-4\vbar/3}
    \setlength{\columnB}{\columnB-2\tabcolsep-4\vbar/3}
    \setlength{\columnC}{\columnC-2\tabcolsep-4\vbar/3}
    
    \begin{longtable}%
        {|\CC{\columnA}|%
          \CC{\columnB}|%
          \LC{\columnC}|%
        }
        \caption[\LaTeX{} text formatting (shape)]{\LaTeX{} text formatting (shape).}%
        \label{tab:tutorial/latex/text/format/shape}\\
        
        \hline
        \multicolumn{1}{|\CC{\columnA}|}{\textbf{\Gls{command}}}
            &\multicolumn{1}{\CC{\columnB}|}{\textbf{Example}}
            &\multicolumn{1}{\CC{\columnC}|}{\textbf{Description}}
        \\\hline
        \endfirsthead
        
        \hline
        \multicolumn{1}{|\CC{\columnA}|}{\textbf{\Gls{command}}}
            &\multicolumn{1}{\CC{\columnB}|}{\textbf{Example}}
            &\multicolumn{1}{\CC{\columnC}|}{\textbf{Description}}
        \\\hline
        \endhead
        
        \texttt{\textbackslash{}textit\{\}}
            &\textit{Text}
            &Italic text.
             \newline -- Shortcut: {\small \tcbox{\texttt{Ctrl}} + \tcbox{\texttt{I}}}
        \\\hline
        
        \texttt{\textbackslash{}textsl\{\}}
            &\textsl{Text}
            &Slanted text.
        \\\hline
        
        \texttt{\textbackslash{}textup\{\}}
            &\textup{Text}
            &Upright text.
        \\\hline
        
        \texttt{\textbackslash{}textsc\{\}}
            &\textsc{Text}
            &Small uppercase text.
        \\\hline
    \end{longtable}
\endgroup

\begingroup
    \setlength{\columnA}{\dimexpr .30\linewidth}
    \setlength{\columnB}{\dimexpr .13\linewidth}
    \setlength{\columnC}{\dimexpr \linewidth-\columnA-\columnB}
    
    \setlength{\columnA}{\columnA-2\tabcolsep-4\vbar/3}
    \setlength{\columnB}{\columnB-2\tabcolsep-4\vbar/3}
    \setlength{\columnC}{\columnC-2\tabcolsep-4\vbar/3}
    
    \begin{longtable}%
        {|\CC{\columnA}|%
          \CC{\columnB}|%
          \LC{\columnC}|%
        }
        \caption[\LaTeX{} text formatting (other)]{\LaTeX{} text formatting (other).}%
        \label{tab:tutorial/latex/text/format/other}\\
        
        \hline
        \multicolumn{1}{|\CC{\columnA}|}{\textbf{\Gls{command}}}
            &\multicolumn{1}{\CC{\columnB}|}{\textbf{Example}}
            &\multicolumn{1}{\CC{\columnC}|}{\textbf{Description}}
        \\\hline
        \endfirsthead
        
        \hline
        \multicolumn{1}{|\CC{\columnA}|}{\textbf{\Gls{command}}}
            &\multicolumn{1}{\CC{\columnB}|}{\textbf{Example}}
            &\multicolumn{1}{\CC{\columnC}|}{\textbf{Description}}
        \\\hline
        \endhead
        
        \texttt{\textbackslash{}emph\{\}}
            &\emph{Text}
            &Emphasized text. This \gls{command} is context aware: if used in normal font text, it will default to italic, and vice-versa.
        \\\hline
        
        \texttt{\textbackslash{}underline\{\}}
            &\underline{Text}
            &Underlined text. \emph{Not recommended}: the \gls{command} encloses its \gls{argument} in a box, which doesn't allow line breaks.
        \\\hline
        
        \texttt{\textbackslash{}textsubscript\{\}}
            &\textsubscript{Text}
            &Subscript inside text.
        \\\hline
        
        \texttt{\textbackslash{}textsuperscript\{\}}
            &\textsuperscript{Text}
            &Superscript inside text.
        \\\hline
    \end{longtable}
\endgroup