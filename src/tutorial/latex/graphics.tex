There are two main ways to draw graphics directly in your documents: \glsname{PGF}/TikZ or PSTricks. The former is better maintained, easier to learn, and more extendable. The latter is more powerful, but it has a steep learning curve. We will stick with TikZ for these reasons. For more documentation and examples on the PSTricks \glspl{package}, be sure to checkout \citename{book:voss-graphics}{author}'s guide on PostScript graphics \parencite{book:voss-graphics}. Readers beware: it's a great resource that is unfortunately riddled with mistakes.

The use case for TikZ can be so specific, I won't bother trying to list all the different options and \glspl{macro}. Rather, I will point readers towards the \LaTeX{} Wikibooks page---which is an easy way to start---and the \gls{CTAN} documentation \parencites{web:wikibooks-latex-tikz,web:ctan-tikz}.