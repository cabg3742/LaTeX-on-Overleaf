\lstinputlisting[%
    language    = LaTeX,
    firstline   = {17},
    lastline    = {17},
    firstnumber = {17},
    caption     = {[Footnotes settings]Footnotes settings.},
    label       = {lst:tutorial/preamble/footnote}
]{tools/settings.tex}

The \texttt{footmisc} \gls{package} allows us to modify how footnotes are displayed. Some of the \glspl{parameter} can be tweaked within the language setup (see \S~\ref{sec:tutorial/preamble/language/options}), but most can be found here. The \glspl{argument} selected in this document to standardize the formatting for the \citelist{report:udes-writing-guide}{institution} are:
\begin{itemize}
    \item \texttt{multiple}:~adds a separator when adding multiple footnotes at the same spot inside the text;
    \item \texttt{bottom}:~forces footnotes at the bottom of the page;
    \item \texttt{flushmargin}:~sets the footnote marker flush with, but just inside the margin from, the footnote content;
    \item \texttt{hang}:~sets the footnote marker flush with the margin.
\end{itemize}

Loading the \gls{package} before \texttt{babel} (see \S~\ref{sec:tutorial/preamble/language}) suppresses the behavior of the \texttt{multiple} \gls{argument}. Loading it after suppresses the \gls{hyperlink} functionality of footnotes provided by the \texttt{hyperref} \gls{package} and adds a warning about the footnote mark redefinition. No known fix exists, so choose which functionality matters more to you... \textit{or be the one to fix it!}