The format can be understood as a higher-level language to write \TeX{} commands. There is plain \TeX{}, which offers no level of abstraction: you're writing with the same \glspl{macro} that were defined in the 70s and 80s. There is also \LaTeX, which is the most common language and the one used by Overleaf. It was created in the 90s to provide flexibility, extension capabilities, and abstraction from plain \TeX. For instance, when you're starting an \gls{environment}, \LaTeX{} translates your simple \gls{command} into \TeX{} \glspl{macro} for the engine to interpret. There are other formats available, such as Op\TeX, though I am much less familiar with those and would prefer not to misguide you in this document. You can find more information about Op\TeX{} in its \gls{CTAN} documentation \parencite{web:ctan-optex}.